\documentclass{article}
\bibliographystyle{ieeetran}
% The first of these two packages allow for accented characters to be copied
% from the resulting PDF properly, and the second package allows for accented
% characters to be entered in the TeX file
\usepackage[T1]{fontenc}
\usepackage[utf8]{inputenc}

% Clickable hyperlinks in the report with \url
\usepackage[hyphens]{url}

% Makes it easy to refer to listings, figures, and the likes of them with
% \autoref{labelName}
\usepackage{hyperref}

% Enable to insert images
\usepackage{graphicx} 

% The listings package is useful for including source code in the report.
% It provides (among other things):
%   * lstlisting environment for listing code directly
%   * lstinputlistings command for listing code from a file
%   * lstinline command for code appearing in the middle of a sentence
\usepackage{listings}
\lstset{%
  language=Bash,
  frame=single,
  numbers=left,
%  basicstyle=\ttfamily,
  breaklines=true,
  columns=fullflexible
}


\usepackage{xcolor}

\usepackage{verbatim}

% Increases the chance that things will fit on a line properly :)
\usepackage{fullpage}

% Make enumerate go (a), (b), ... instead of 1., 2., ...
\renewcommand{\labelenumi}{(\alph{enumi})}

% ... and then I., II., ... instead of (a), (b), ...
\renewcommand{\labelenumii}{\Roman{enumii}.}

% The enumerate package and \begin{enumerate}[(a)] or the likes of it can be
% used to break the norm though
\usepackage{enumerate}

% Avoids "code" duplication
\newcommand{\highergradesonly}{[\textbf{higher grades only}]}

% Possibly useful things for making the text more readable
\newcommand{\program}[1]{#1}
\newcommand{\service}[1]{#1}
\newcommand{\command}[1]{#1}
\newcommand{\code}[1]{#1}
\newcommand{\flag}[1]{#1}
\newcommand{\myurl}[1]{#1}
\newcommand{\user}[1]{#1}
\newcommand{\password}[1]{#1}
\newcommand{\directory}[1]{#1}
\newcommand{\file}[1]{#1}
\newcommand{\parameter}[1]{#1}
\newcommand{\exploitname}[1]{\textit{#1}}
\newcommand{\msexploit}[1]{#1}
\newcommand{\option}[1]{#1}
\newcommand{\payloadname}[1]{#1}
\newcommand{\ipaddress}[1]{#1}
\newcommand{\ip}[1]{\ipaddress{#1}}
\newcommand{\filename}[1]{#1}

\title{\textbf{Secure Computer Systems I: Group Assignment}}
\author{Ren Li \and Tianyao Ma \and Samuel Pettersson}

\begin{document}
\maketitle

\begin{enumerate}
\item
There are a few different aspects of surveillance that we feel are determining factors of whether the surveillance is acceptable.
\begin{itemize}
\item
The first aspect is anonymity---whether or not the surveilled individual can be identified. Surveillance through which the individuals cannot be identified is of course preferable, but anonymity is not a sufficient condition for acceptability.
\item
The second aspect is whether or not the surveillance is carried out in secrecy or not. Knowing that you are being surveilled is mostly a good thing (ignorance is a bliss to some extent though). One way of obtaining such knowledge is having the surveiller explicitly tell you that information is collected. We argue that there is difference of some importance between ``hiding'' surveillance statements in the middle of terms of agreements and being really open about it.
\item
A third aspect, mostly related to digital surveillance, is whether just metadata---data about communication taking between two entities taking place---or actual data, too, is collected. Superficially, collection of metadata does not appear to be particularly harmful, but we do not find it obvious to be harmless. It is also worthy to note that, as an end-user, it is nigh impossible to determine whether just metadata or actual data, too, is collected by, for instance, the ISP.
\end{itemize}
We think that the first aspect is the most important one: surveillance that respects the anonymity of the individuals is in general acceptable, mostly because it has no obvious negative consequences. Unfortunately, some surveillance do not fall into said category; one such example is the targeted advertisements by Google.
\item
Companies typically perform surveillance for economic reasons. For instance, collecting information about their users, Google can target their advertisements. Governments, on the other hand, may perform surveillance for a variety of reasons of different levels of legitimacy. They might collect information about their citizens to make more appropriate plans for the future or to catch criminals. However, there are also indications that there are other, perhaps more iniquitous reasons behind their surveillance. Our opinions differ slightly on which source of surveillance is the most acceptable (least unacceptable): one of us believes that surveillance by companies is more acceptable, at least superficially, seeing as the companies might have less of a hidden agenda, whereas another opinion is that non-commercial surveillance is more acceptable than the commercial counterpart. An example of why commercial surveillance could be just as bad as that carried out by governments is when companies sell the collected information to the government.


\end{enumerate}
\bibliography{lab3}
\end{document}
