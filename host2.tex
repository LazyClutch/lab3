\section*{Host 2: 192.168.233.110}
The second host to which root access was gained was \ip{192.168.233.110}. The instructions said that access to host \ip{192.168.233.20} was required to access this machine.

Running \program{nmap} showed that two services were running on the target machine: \service{Microsoft Windows RPC} on port 135 and \service{Microsoft Terminal Service} on port 3389, the latter being used for serving remote desktop clients. The \program{nmap} command and its output is shown below.
\verbatiminput{nmap_110.txt}
The \flag{Pn} flag is used to skip verifying that the host is online with ping probes; the target seemed to block them. The \flag{v} and \flag{A} flags are for verbose output and various types of version detection.

With no obvious vulnerabilities being present in the RPC service, attention was turned to the remote desktop service. The RDP client of our choice was (initially) \program{rdesktop}, which is available in the software package repository and was installed by issuing the command \command{sudo apt-get install rdesktop}. Connecting to the other machine was as simple as running the command \command{rdesktop 192.168.233.110}. Upon doing so, we were presented with the Modern UI login screen, characteristic of Windows 8. Authentication as the administrator appeared to require a smart card, whereas password authentication was an option for other users.

After some tinkering with a remote desktop connection on host \ip{192.168.233.20}, we found out that password authentication as the administrator on \ip{192.168.233.110} was possible by starting a remote desktop session from \ip{192.168.233.20} to \ip{192.168.233.110}. One could suspect that the administrator password for the two machines were the same, but despite having root access to \ip{192.168.233.20}, we were unaware of what the password to the administrator account was.

The immediate thought was to dump the password hashes on host \ip{192.168.233.20} and have them cracked either by brute force or with a dictionary. Dumping the hashes is simple enough with a meterpreter shell: issuing the command \command{run hashdump} does the trick. For each user, a composite NTLM hash comprising two hashes were stored: an unused LM hash of the empty string and an NT hash\cite{wikipedia_NTLM}. Unfortunately, the version of John the Ripper available on the BackBox machine did not have support for breaking NT hashes. The source code of the latest ``jumbo'' version of John the Ripper (version 1.7.9-7) with support for NT hashes was downloaded from \url{http://www.openwall.com/john/}. Building the application was a matter of running two \program{make} commands: \command{make} and \command{make clean linux-x86-mmx} as per the README and the installation instructions. After having run the jumbo version of John the Ripper for several hours without finding a password matching any of the dumped NT hashes, let alone the hash for the administrator account, the trail grew colder and we gave up on retrieving the password.

After having received a hint from Sofia, we decided to look into the possibility of passing the hash of the administrator on \ip{192.168.233.20} (CL3\textbackslash Administrator). That is, instead of authenticating with a password, we hoped to be able to authenticate with the hash of the password. Some research showed that Microsoft's Remote Desktop Protocol version 8.1 from last year introduced a Restricted Admin mode, in which credentials in the form of passwords are not sent to the remote server; instead, password hashes are sent. This mode was introduced in an effort to improve security in that authentication to a compromised server does not reveal the password in cleartext, but unfortunately, it comes at the expense of allowing pass the hash attacks\cite{portcullis_blog}. Further research showed that an RDP client by the name FreeRDP with support for passing hashes was available in the form of a Git repository at GitHub\cite{portcullis_tools}\cite{freerdp_git_repo}.

The repository was cloned and the application was built by following the installation instructions available at GitHub\cite{freerdp_installation_instructions}. The necessary packages were installed, \program{cmake} was used to generate Makefiles, and \program{make} was used to build and install the application. After having completed the installation, the client was available by the name \program{xfreerdp}.

The following command was executed to attempt to log on as the administrator with the NT hash (the string starting with ``fabc'') retrieved from host \ip{192.168.233.20}:\\
\command{xfreerdp /u:Administrator /pth:fabc417905666832e4b4ba57711a4171 /v:192.168.233.110}.\\Passing the hash turned out to work! A text file containing the secret code without any encryption was easily spotted on the desktop; the code to include in the report read: \code{vZxkRvEICzhNQ}.
